% das Papierformat zuerst
\documentclass[a4paper, 11pt]{article}

% deutsche Silbentrennung
\usepackage[ngerman]{babel}

% wegen deutschen Umlauten
\usepackage[utf8]{inputenc}

% hier beginnt das Dokument
\begin{document}
\maketitle
\section{Abschnitt}
\label{sec:abschnitt}

\subsection{Unterabschnitt}
\label{subsec:unterabschnitt}

Einzelne Worte kann man \emph{hervorheben}. Man sollte dieses Mittel aber sehr
sparsam einsetzen. In diesem Beispiel soll es aber um
Fußnoten\footnote{\label{foot:1}Dies ist jetzt eine Fußnote.} gehen.

% neue Seite
\newpage

\section{Weiterer Abschnitt}

\subsection{Weiterer Unterabschnitt}

Weiterhin soll in diesem Beispiel erwähnt sein, wie man Verweise auf andere
Abschnitte erstellt, also z.~B. auf Abschnitt \ref{sec:abschnitt} auf Seite
\pageref{sec:abschnitt}. Dabei ist es natürlich wichtig, dass man die Labels
richtig setzt. So gibt es natürlich auch noch
Unterabschnitt\footnote{\label{foot:2}Und dies ist natürlich auch eine Fußnote.}
\ref{subsec:unterabschnitt} auf Seite \pageref{subsec:unterabschnitt}. Man kann
natürlich auch auf Fußnoten wie Fußnote \ref{foot:1} auf Seite \pageref{foot:1}
und Fußnote \ref{foot:2} auf Seite \pageref{foot:2} verweisen.

% das ist wohl jetzt das Ende des Dokumentes
\end{document}
