% das Papierformat zuerst
\documentclass[a4paper, 11pt]{article}

\usepackage[utf8]{inputenc}

% deutsche Silbentrennung
\usepackage[ngerman]{babel}

% wir wollen auf jeder Seite eine Ueberschrift
\pagestyle{plain}

% hier beginnt das Dokument
\begin{document}

% Inhaltsverzeichnis anzeigen
\tableofcontents

% Kapitel soll auf naechster Seite beginnen
\newpage

% Kapitelueberschrift
\section{Einleitung}

% Ueberschrift eines Abschnittes
\subsection{Motivation}

Ein bekannter Stolperstein bei Latex sind die deutschen Umlaute wie: ä, ö, ü,
Ä, Ö, Ü und natürlich auch das ß.

% Kapitel soll auf naechster Seite beginnen
\newpage

% Kapitelueberschrift
\section{Theoretische Betrachtungen}

Dies ist ein Satz.

% das ist wohl jetzt das Ende des Dokumentes
\end{document}
